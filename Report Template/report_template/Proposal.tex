% LLNCStmpl.tex
% Template file to use for LLNCS papers prepared in LaTeX
%websites for more information: http://www.springer.com
%http://www.springer.com/lncs

\documentclass{llncs}
%Use this line instead if you want to use running heads (i.e. headers on each page):
%\documentclass[runningheads]{llncs}


\begin{document}
\title{Linux client with WiFi and LTE}

%If you're using runningheads you can add an abreviated title for the running head on odd pages using the following
%\titlerunning{abreviated title goes here}
%and an alternative title for the table of contents:
%\toctitle{table of contents title}

\subtitle{Project proposal}

%For a single author
%\author{Author Name}

%For multiple authors:
\author{Pascal Maissen, Jovana Micic, Noe Wysshaar} 


%If using runnningheads you can abbreviate the author name on even pages:
%\authorrunning{abbreviated author name}
%and you can change the author name in the table of contents
%\tocauthor{enhanced author name}

%For a single institute
%\institute{Institute Name \email{email address}}

% If authors are from different institutes 
\institute{University of Bern\\  \email {pascal.maissen@unifr.ch, jovana.micic@students.unibe.ch, noemathieu.wysshaar@unifr.ch} }


%to remove your email just remove '\email{email address}'
% you can also remove the thanks footnote by removing '\thanks{Thank you to...}'


\maketitle

%\begin{abstract}
%abstract text goes here - Lorem ipsum dolor sit amet, consectetur adipiscing elit, sed do eiusmod tempor incididunt ut labore et dolore magna aliqua.
%\end{abstract}

\section{Project description}
The task of this project is study of Dynamic Adaptive Streaming over HTTP video delivery in a mobile scenario using WiFi and Long Term Evolution (LTE). Case scenario is that the user is connected through LTE, but periodically gets the internet access through WiFi. The main goal of this project is to evaluate the quality gain in the parallel LTE/WiFi video transmission in comparison to a single LTE transmission using multi-path TCP (MPTCP).\\ \\
\textbf{Key Performance Indicators (KPIs)} will be the following:
\begin{itemize}
\item \textbf{Bit Rate:} Measuring the bitrate differences on LTE vs. LTE and WiFi to define quality of the video stream.
\item \textbf{Buffering:} How big needs the buffer to be to ensure no waiting time for loading the video. Fill the buffer up when WiFi is available.
\item \textbf{Lag:} Keeping the lag at a minimum, especially when streaming a live event where the buffer should be small to reduce overall lag.
\end{itemize}


\section{Roadmap of project}
Idea is to study the topic using MPTCP on Linux. User equipment (UE) will be connected to LTE and it will periodically get access to WiFi. First step will be establishing experimental setup on several laptops and PCs. Then we will start downloading video from some video server using only LTE. While the video is downloading, we will turn on and off WiFi. When WiFi access point is on we expect better experience in comparison to LTE.  In next steps we will study quality gain in different scenarios. 


\section{Explanation of concepts}
\textbf{Long Term Evolution (LTE)} is standard for high-speed wireless communication for mobile devices. It increases capacity and speed using new digital signal processing techniques. It represents 4th generation technology. \\  \\
\textbf{Wireless fidelity (WiFi)} is technology for wireless local area networking (WLAN) with devices based on the IEEE 802.11 standards. Devices compatible with WiFi can connect to the Internet using WLAN and a wireless access point. Access points have range of about 20 meters indoors. \\ \\
\textbf{Dynamic Adaptive Streaming over HTTP (DASH) } is streaming technique that enables high quality streaming of media content over the Internet. Client automatically selects the next segment to download based on current network conditions. It can adapt to changing network conditions.\\  \\
\textbf{Multi-path TCP (MPTCP)} is solution that allows applications to use multiple paths for streaming. This feature is common on today's laptops and smartphones. Using MPTCP dramatically improves Quality of Experience of video streaming. Streaming video over MPTCP may incur undesired network usage. 

\section{Milestones}
\begin{center}
\begin{tabular}{ | r | l |} \hline
March 2018 & Start of project \\ \hline
Early April 2018 & Implementation \& first draft version of report \\ \hline
Late April 2018 & Finishing implementation \\ \hline
30. April 2018 & Midterm presentation \\ \hline
Early May 2018 & Evaluation \& Analysis of results, writing report \\ \hline
Late May 2018 & Finishing up \\ \hline
28. May 2018 & Final project presentation \\ \hline

\end{tabular}
\end{center}


%The bibliography, done here without a bib file
%This is the old BibTeX style for use with llncs.cls
\bibliographystyle{splncs}

%Alternative bibliography styles:
%the following does the same as above except with alphabetic sorting
%\bibliographystyle{splncs_srt}
%the following is the current LNCS BibTex with alphabetic sorting
%\bibliographystyle{splncs03}
%If you want to use a different BibTex style include [oribibl] in the document class line

\begin{thebibliography}{1}
%add each reference in here like this:
\bibitem[RE1]{reference1}
Bo Han, Feng Qian, Lusheng Ji, and Vijay Gopalakrishnan. 2016. MP-DASH: Adaptive Video Streaming Over Preference-Aware Multipath. In Proceedings of the 12th International on Conference on Emerging Networking EXperiments and Technologies (CoNEXT ’16). ACM, New York, NY, USA, 129–143.

\bibitem[RE2]{reference2}
Thomas Stockhammer. 2011. Dynamic Adaptive Streaming over HTTP – Design Principles and Standards. MMSys '11 Proceedings of the second annual ACM conference on Multimedia systems. San Jose, CA, USA .133-144.


\end{thebibliography}

\end{document}

